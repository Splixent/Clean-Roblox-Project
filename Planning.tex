\documentclass[11pt]{article}
\usepackage{enumitem}
\usepackage{geometry}
\usepackage{xcolor}
\usepackage{sectsty}
\usepackage{tcolorbox}
\usepackage{titlesec}
\usepackage{helvet}
\renewcommand{\familydefault}{\sfdefault}
\geometry{margin=0.75in}
\setlist[itemize]{topsep=2pt, itemsep=1pt, parsep=0pt, leftmargin=*}
\titlespacing*{\section}{0pt}{8pt}{4pt}
\titlespacing*{\subsection}{0pt}{6pt}{3pt}
\titlespacing*{\subsubsection}{0pt}{4pt}{2pt}
\titlespacing*{\paragraph}{0pt}{3pt}{2pt}

% Define color scheme
\definecolor{terracotta}{RGB}{191, 85, 60}
\definecolor{clay}{RGB}{71, 85, 105}
\definecolor{lightclay}{RGB}{148, 163, 184}
\definecolor{glaze}{RGB}{59, 130, 246}

% Section styling
\sectionfont{\color{terracotta}\Large}
\subsectionfont{\color{clay}\large}
\subsubsectionfont{\color{glaze}\normalsize}

% Title styling
\title{\color{terracotta}\textbf{Pottery Game - Planning Document}}
\author{}
\date{}

\begin{document}

\maketitle

\section{Game Mechanics}

\subsection{Plot}

\subsubsection{Clay Patch}
\begin{itemize}
    \item Pick up normal clay from this area to use to make pottery
\end{itemize}

\subsubsection{Potter's Wheel}
\begin{itemize}
    \item Use your clay to create pottery
    \item Choose what kind of pottery to make from your Potter's Book
\end{itemize}

\subsubsection{Glaze Table}
\begin{itemize}
    \item \textit{[Not sure how this should work completely]}
    \item Unlocked after completing the tutorial
    \item Choose what kind of glaze your pottery should have before firing
\end{itemize}

\subsubsection{Kiln}
\begin{itemize}
    \item Used to fire your pottery
\end{itemize}

\subsubsection{Cooling Table}
\begin{itemize}
    \item Pottery must cool down here after firing in the kiln
    \item Required step before pottery can be moved to showroom
\end{itemize}

\subsubsection{Showroom}
\begin{itemize}
    \item Where completed pottery goes to be purchased by NPCs
\end{itemize}

\subsection{Inventory System}

\subsubsection{Hotbar}
\begin{itemize}
    \item No traditional inventory - just a hotbar for simplicity (2-week project scope)
    \item Clay stacks in hotbar slots when collected from Clay Patch
    \item Future: different clay types stack separately
    \item Future: misc tools in hotbar (heat gloves, glaze tongs, etc.)
\end{itemize}

\subsubsection{Clay Usage at Potter's Wheel}
\begin{itemize}
    \item Select a pottery style from the Potter's Book while at the wheel
    \item A semi-transparent preview model of the pottery appears on the wheel
    \item Indicator shows how much clay is required for the selected style
    \item Click the wheel with stacked clay in hand to deposit clay
    \item Once enough clay is deposited, the forming minigame begins
    \item Different pottery styles require different clay amounts
    \item Future: some pottery requires specific clay types
\end{itemize}

\subsubsection{Potter's Wheel Minigame}
\begin{itemize}
    \item \textbf{Rhythm-based:} Timed button presses as the wheel spins - hit markers as they pass through a zone
    \item \textbf{Why rhythm works:} Contrasts with spam-click harvesting, feels more skilled/satisfying, matches the spinning wheel aesthetic
    \item Success quality affects pottery value/appearance
    \item Higher level pottery = faster/more complex patterns
\end{itemize}

\subsection{Potter Level / Achievements}
\begin{itemize}
    \item Potter Level increases as you complete achievements in your Achievements Book
    \item Some Pottery Styles and Glazes are locked behind Level requirements
\end{itemize}

\subsection{Plot Upgrades}

\subsubsection{Pottery Station Upgrades}
\begin{itemize}
    \item All Stations start at level 0
    \item Upgrades require a minimum level \& Nyra (₦)
\end{itemize}

\paragraph{Clay Patch Upgrades}
\begin{itemize}
    \item More clay per harvest
    \item Faster replenish time
    \item Higher clay storage cap
\end{itemize}

\paragraph{Potter's Wheel Upgrades}
\begin{itemize}
    \item Faster shaping time
    \item Unlocks Pottery Styles at certain levels
\end{itemize}

\paragraph{Glaze Table Upgrades}
\begin{itemize}
    \item \textit{[To be determined]}
\end{itemize}

\paragraph{Kiln Upgrades}
\begin{itemize}
    \item Faster firing time
    \item Larger batch capacity
\end{itemize}

\paragraph{Cooling Table Upgrades}
\begin{itemize}
    \item Faster cooling time
    \item Increased cooling slots
\end{itemize}

\subsubsection{Plot Theme Upgrades}
\begin{itemize}
    \item Plot starts small and simple with dirt floor
    \item Unlock cosmetic plot themes that transform the entire workspace aesthetic
    \item All themes accommodate all crafting stations (Clay Patch, Potter's Wheel, Glaze Table, Kiln, Cooling Table, Showroom)
    \item Each theme features a unique showroom design - the primary visual flex for other players
    \item Example themes:
    \begin{itemize}
        \item Pink Pottery Factory - cute, modern aesthetic
        \item Industrial Workshop - gritty, factory vibe
        \item \textit{[More themes to be designed]}
    \end{itemize}
    \item Themes are primarily cosmetic but serve as status symbols to show off progression
\end{itemize}

\section{UI Style}
\textit{[To be determined]}

\section{Update Schedule}
\textit{[To be determined]}

\end{document}
